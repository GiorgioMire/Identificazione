\documentclass[10pt,a4paper,oneside,openany]{book}
\usepackage[utf8]{inputenc}

\usepackage{amsmath}
\usepackage{amsfonts}
\usepackage{amssymb}
\usepackage{graphicx}

\begin{document}
\chapter{Introduzione}
\begin{minipage}{0.5\textwidth}
\noindent
The number of things you don't know is one of the things you don't know\vspace{1em}\\Peter Green
\end{minipage}\vspace{3em}\\
Quando si vuole identificare un sistema dinamico una fase importante è stabilire la struttura del modello. Spesso infatti non sono incognite solo le costanti della legge che regola il processo ma non si conoscono nemmeno quali operazioni sui segnali di ingresso e di stato riproducono accettabilmente la dinamica del sistema.\\
I casi in cui è sufficiente una identificazione parametrica sono i casi in cui si ha già la struttura della legge. Per ottenere la legge (a meno del valore delle costanti) è necessario uno studio del sistema a partire dai principi primi che regolano tutti i fenomeni che sono coinvolti. In definitiva l'idengtificazione parametrica è sufficiente solo se sono verificate le seguenti condizioni
\begin{itemize}
\item è \textbf{possibile} ottenere una descrizione del sistema applicando i principi primi delle scienze
\item è \textbf{vantaggioso} ottenere una descrizione del sistema applicando i principi primi delle scienze (tempo speso, rischio di errori etc)
\item ci si accontenta della descrizione ottenuta applicando i principi primi delle scienze (rischio di tralasciare fenomeni importanti)
\item la complessità delle leggi che si ottengono applicando i principi primi delle scienze non è tale da renderle inutilizzabili
\end{itemize}
Quando una di queste condizioni non è soddisfatta è possibile inserire nel processo di identificazione anche la ricerca della struttura del modello.
A tale scopo si ipotizza una classe (abbastanza vasta) di strutture di modello
e si lascia al processo di identificazione l’onere di scegliere quella che maggiormente si adatta
alle misure. I metodi Bayesiani in particolare sono appropriati per la selezione dei
modelli perchè forniscono naturalmente (senza analisi aggiuntve) un contesto in cui
si quantifica l’incertezza associata all’identificazione. L’approccio bayesiano infatti, non fornisce una singola descrizione del sistema ma fornisce una distribuzione di
probabilità associata al set di modelli possibili. Dunque risolvere il problema di
inferenza bayesiana conduce a scelte più  ponderate rispetto alla semplice estrazione
del modello che si adatta meglio ai dati reali. Analizzando la distribuzione prodotta
si può per esempio individuare un modello poco meno accurato del migliore ma
preferirlo perchè meno complesso .\\
Nella seguente trattazione la classe di modelli scelta è quella dei nonlineari, auto-regressivi, a media mobile
con ingressi esogeni (NARMAX); essa è una popolare classe modelli ingresso-uscita
spesso usata nell’identificazione di sistemi non lineari in vari ambiti dell’ingegneria
(e non) poichè ha il pregio di produrre leggi compatte ma accurate.
Come si vedrà nei successivi capitoli, ciascun modello NARMAX può essere visto
come lo sviluppo su una opportuna base polinomiale della legge che
regola il sistema. Ciò che distingue un modello da un altro è il numero e l’insieme
dei termini presenti nello sviluppo.

La ricerca esaustiva tra tutte le possibili combinazioni di termini NARMAX è tuttavia quasi sempre computazionalmente improponibile , così si è puntato negli ultimi
anni a sviluppare metodi di campionamento efficaci che, avvalendosi di una catena
di Markov opportunamente costruita, esplorano in maniera non esaustiva l’insieme
dei modelli ottenendo comunque statistiche significative.
\chapter{Identificazione bayesiana e metodi
di campionamento Monte Carlo}
\section{Approccio Bayesiano per l’identificazione}
L’inferenza bayesiana è un approccio all’inferenza statistica in cui le probabilità
non sono interpretate come frequenze ma piuttosto come livelli di fiducia nel verificarsi di un dato evento. Il nome deriva dal teorema di Bayes, che costituisce il
fondamento di questo approccio. Gli statistici bayesiani sostengono che i metodi
dell’inferenza bayesiana rappresentano una formalizzazione del metodo scientifico,
che normalmente implica la raccolta di dati che avvalorano o confutano una data
ipotesi.
Queste caratteristiche rendono l’approccio bayesiano un utile ausilio per discriminare tra alternative in conflitto e dunque un ottimo strumento per l’identificazione
dei sistemi.
Il metodo usa una stima del grado di fiducia in una data ipotesi prima dell’osservazione
dei dati al fine di associare un valore numerico al grado di fiducia in quella stessa
ipotesi successivamente all’osservazione dei dati. Si supponga di voler identificare
un sistema scegliendo il più adatto in un insieme M di modelli.
Sia M una variabile aleatoria che rappresenta la legge che regola il vero sistema.
L’obiettivo dell’identificazione bayesiana è quello di calcolare per ogni $m\in M$ la
probabilità a posteriori
\[P (M = m|Y )\]
dunque complessivamente determinare la distribuzione della probabilità sull’insieme
dei modelli condizionatamente alle misure.
La probabilità condizionata è stata definita in termini della probabilità congiunta e
marginale dei due eventi
\[P (M = m|Y ) =
P (M = m, Y )
P (Y )\]
la definizione corrisponde all’idea ragionevole
\[P (M = m|Y ) \propto P (M = m, Y )\]
Avendo per’inferenza bayesiana è un approccio all’inferenza statistica in cui le probabilit`a
non sono interpretate come frequenze ma piuttosto come livelli di fiducia nel ver-
ificarsi di un dato evento. Il nome deriva dal teorema di Bayes, che costituisce il
fondamento di questo approccio. Gli statistici bayesiani sostengono che i metodi
dell’inferenza bayesiana rappresentano una formalizzazione del metodo scientifico,
che normalmente implica la raccolta di dati che avvalorano o confutano una data
ipotesi.
Queste caratteristiche rendono l’approccio bayesiano un utile ausilio per discrim-
inare tra alternative in conflitto e dunque un ottimo strumento per l’identificazione
dei sistemi.
Il metodo usa una stima del grado di fiducia in una data ipotesi prima dell’osservazione
dei dati al fine di associare un valore numerico al grado di fiducia in quella stessa
ipotesi successivamente all’osservazione dei dati. Si supponga di voler identificare
un sistema scegliendo il pi`
u adatto in un insieme M di modelli.
Sia M una variabile aleatoria che rappresenta la legge che regola il vero sistema.
L’obiettivo dell’identificazione bayesiana è quello di calcolare per ogni $m\in M$ la
probabilit`a a posteriori
\[P (M = m|Y )\]
dunque complessivamente determinare la distribuzione della probabilit`a sull’insieme
dei modelli condizionatamente alle misure.
La probabilit`a condizionata è stata definita in termini della probabilit`a congiunta e
marginale dei due eventi
\[P (M = m|Y ) =
P (M = m, Y )
P (Y )\]
la definizione corrisponde all’idea ragionevole
\[P (M = m|Y ) \propto P (M = m, Y )\]
Avendo però ristretto l’insieme di supporto su cui è definita la metrica di probabilit`a,
è necessario imporre che valgano ancora gli assiomi della probabilit`a, in particolare
chiamando k la costante di proporzionalit`a e integrando su tutti i casi possibili si ha
\[P (M = m|Y )dm =
k \cdot P (M = m, Y )dm = k \cdot P (Y ) := 1\]
da cui si ricava il valore di k
\[k =
\frac{1}{P(Y)}\]
ottenendo la definizione di probabilità condizionata.
Scambiando i ruoli delle variabili aleatorie si ha anche che
P (Y |M = m) =
P (M = m, Y )
P (M = m)
(2.2)
dunque mettendo insieme le equazioni (2.1) e la (2.2) si ottiene la formula di Bayes
P (M = m|Y ) =
P (Y |M = m)P (M = m)
P (Y )
(2.3)
usando il teorema della probabilit`a totale si pu`o inoltre esprimere la costante al
denominatore in funzione delle quantit`a al numeratore ottenendo
P (M = m|Y ) =
P (Y |M = m)P (M = m)
P (Y |M = m)P (M = m)dm
(2.4)
Nei problemi di identificazione l’espressione della probabilit`a a posteriori è quasi
sempre impossibile da valutare analiticamente sopratutto per colpa dell’integrale al
denominatore che deve essere calcolato su tutti i possibili modelli; si ricorre quindi a
metodi numerici di tipo Monte Carlo basati sul campionamento della distribuzione.
Spesso risulta impossibile anche campionare direttamente la distribuzione e se ne
deve ottenere una stima accettando o scartando opportunamente i campioni estratti
da una distribuzione pi`
u semplice.
\end{document}