\chapter{Reversible Jump Monte Carlo Markov Chain}
Nelle sezioni precedenti la transizione della catena associava (in maniera aleatoria)
uno stato di $M\subset \mathbb{R}^n$ ad uno stato di  $M'\subset \mathbb{R}^n$ .
Cosa succede se la transizione dovesse associare uno stato $M\subset \mathbb{R}^n$ ad uno stato
di $M\subset \mathbb{R}^m$ con $ m = n$?
La questione scaturisce dal fatto che per gli scopi dell’identificazione dei sistemi,
restringersi al caso $m = n$ `e molto limitativo e corrisponde a conoscere in partenza
il numero dei parametri da identificare.
Per i sistemi non lineari spesso si considerano modelli del sistema che sono sviluppi
polinomiali di equazioni alle differenze e si lascia all’algoritmo di identificazione
l’onere di determinare quali e quanti termini dello sviluppo includere.
In base al numero di termini scelti si avranno da scegliere anche altrettanti coeffici-
enti dello sviluppo.
Quando l’algoritmo suggerisca di aggiungere o eliminare uno (o pi`u) termini dello
sviluppo `e necessario aggiornare anche la dimensione del vettore dei coefficienti.
Ecco quindi che acquistano senso mosse che portano lo stato della catena da uno
spazio con una certa dimensionalit`a ad un’altro con dimensionalit`a diversa.
Si pensi quindi di enumerare le tipologie di modello (anche infinite), si avr`a che
l’insieme delle possibili strutture di modello `e rappresentabile come un insieme di
indici
\begin{equation*}
K=\left\lbrace 1,2,\dots k \right\rbrace
\end{equation*}