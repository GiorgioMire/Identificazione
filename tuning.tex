\chapter{Tuning degli iperparametri}
Il tuning degli iperparametri viene fatto basandosi sulle espressioni analitiche di valor medio e varianza della distribuzione.\\
Nel caso della variabile Gamma che modella il valore atteso del numero di termini si ha il seguente valor medio
\begin{equation}
\mathbb{E}[\mathcal{G}\mathcal{A}(\alpha,\beta)]=\frac{\alpha}{\beta}:=m
\end{equation}
e la seguente varianza
\begin{equation}
\mathbb{V}ar[\mathcal{G}\mathcal{A}(\alpha,\beta)]=\frac{\alpha}{\beta^2}:=v
\end{equation}
da cui
\begin{equation}
\alpha=m\beta
\end{equation}
\begin{equation}
\alpha=v\beta^2
\end{equation}
dunque
\begin{equation}
v\beta^2=m\beta
\end{equation}
\begin{equation}
(v\beta-m)\beta=0
\end{equation}
dato che il parametro $beta$ non può essere nullo si giunge infine alle relazioni inverse
\begin{equation}
\begin{cases}
\beta=\frac{m}{v}\\
\alpha=\frac{m^2}{v}
\end{cases}
\end{equation}
Si noti che quando si condiziona rispetto al numero di termini $k$ (oppure $q$) esso si va ad aggiungere al parametro $\alpha$
dunque le espressioni corrette per imporre la varianza sono
\begin{equation}
\begin{cases}
\beta=\frac{m}{v}\\
\alpha=\frac{m^2}{v}-k
\end{cases}
\end{equation}
Nemmeno il parametro $\alpha$ però può essere negativo, dunque si ha che il valor medio e la varianza non possono essere scelti arbitrariamente ma deve valere
\begin{equation}
\frac{m^2}{v}>k
\end{equation}

mentre la distribuzione gamma inversa ha per valor medio
\begin{equation}
\mathbb{E}[\mathcal{I}\mathcal{G}(\alpha,\beta)]=\frac{\beta}{\alpha-1}
\end{equation}
e varianza
\begin{equation}
\mathbb{V}ar[\mathcal{I}\mathcal{G}(\alpha,\beta)]=\frac{\beta^2}{(\alpha-1)^2(\alpha-2)}
\end{equation}