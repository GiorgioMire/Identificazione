\chapter{Algoritmo Metropolis Hasting}
Supponiamo di voler estrarre campioni da una distribuzione p(m) nota, difficile da
campionare direttamente.
Si pu`o chiedere che la distribuzione obiettivo sia
distribuzione di regime di una catena di Markov. Una catena si dice ergodica se
dopo un certo tempo (transitorio iniziale) , essa converge all'unica distribuzione
di regime indipendentemente dalla distribuzione di partenza.
Le condizioni necessarie per l’ergodicit`a della catena sono:
\begin{itemize}
\item \textbf{irriducibilit`} : la probabilit`a di visitare ciascuno stato a partire da ciascuno
stato `e strettamente positiva
\item \textbf{aperiodicità} : una catena `e periodica se pu`o ritornare in un certo stato solo
a istanti multipli di un qualche intero maggiore di 1. Una catena `e aperiodica
se non `e periodica.
\end{itemize}

In poche parole, fissato un qualsiasi istante abbastanza grande e un qualsiasi stato,
deve essere possibile una storia temporale della catena che la porta a risiedere in
quello stato a quell’istante. La questione diventa quindi come scegliere il kernel
\begin{equation*}
s(x' |x) 
\end{equation*}
(probabilit`a di transizione dal vecchio stato m al nuovo stato m' ) della catena,
in modo che la catena sia ergodica e che la distribuzione di regime sia proprio p(m).
Una condizione sufficiente non necessaria `e che la distribuzione obiettivo soddisfi la
condizione di reversibilit`a

\begin{equation}
p(x)s(x' |x) = p(x')s(x|x' )
\end{equation}

L'operatore di transizione è quello che nelle catene a stato discreto e finito era rappresentato dalla matrice di transizione mentre nelle catene a stato continuo è il funzionale
\begin{equation}
Tr[\bullet]=\int \bullet(x)s(x'|x)dx
\end{equation}
Una  densit`a di probabilit`a $\ph$ `e detta stazionaria se `e autofunzione
dell’operatore di transizione della catena, ovvero\\
\begin{equation}
Tr[\ph]=\ph
\end{equation}
e semplice dimostrare che se vale la condizione di reversibilit`a allora la distribuzione
`e stazionaria infatti
\begin{equation*}
\begin{split}
Tr[\ph]=\int \ph(x)s(x'|x)dx=\int \ph(x')s(x|x')dx=\\
\ph(x')\int s(x|x')dx=\ph(x')\cdot 1=\ph(x')
\end{split}
\end{equation*}

Se si sceglie un kernel arbitrario uguale ad una probabilit`a di transizione facile
da campionare $s(\direct) = q(\direct) $ solitamente la condizione di reversibilit`a non `e
soddisfatta. Solitamente si ha uno sbilanciamento che significa che alcune transizioni
sono pi`u probabili in un certo verso.
\begin{equation}
\ph(x)s(\direct)>\ph(x')s(\inverse)
\end{equation}

In tal caso, viste le distribuzioni di partenza e la probabilit`a di transizione scelta `e
pi`u probabile osservare la transizione $x\rightarrow x'$ piuttosto che la transizione $x'\rightarrow x$
Si cerca quindi di ristabilire l’equilibrio scegliendo come kernel la probabilit`a di
transizione moltiplicata per un fattore correttivo 
\begin{equation}
q(\direct) = \gamma(\direct)q(\direct)
\end{equation}
In particolare si pu`o interpretare $\gamma(\direct)$ come la probabilit`a di accettare la mossa $x \rightarrow x'$. Imponendo quindi che valga


\begin{equation}
\ph(x)\gamma(\direct)q(\direct)=\ph(x')\gamma(\inverse)q(\inverse)
\end{equation}
si ricava
\begin{equation}
\frac{\gamma(\direct)}{\gamma(\inverse)}=\frac{\ph(x')q(\inverse)}{\ph(x)q(\direct)}
\end{equation}

Perch`e si abbia coerenza con la eq[CITARE EQUAZIONE] bisogna che il rapporto al membro sinistro sia minore di 1. In particolare si può scegliere

\begin{itemize}
\item $\gamma(\inverse)=1$ che equivale ad accettare tutte le transizioni $m' \rightarrow m$ che sono meno
frequenti
\item $\gamma(\direct)=min\left\lbrace 1,\frac{\ph(x')q(\inverse)}{\ph(x)q(\direct)}\right\rbrace$ fattore di riduzione delle transizioni pi`u frequenti
\end{itemize}
\newpage
 L'algoritmo MH è così definito\vspace{1em}\\
\hrule 
\textbf{Algoritmo} Metropolis Hastings
\hrule



\begin{algorithmic}
\State Inizializza  $X_0=x_0$
\For {$t=0,1,2\dots T$}
    \State $i\gets 0$

\State Estraggo la proposta per il nuovo stato $x^t \sim q(x^t |X_t)$
\State Calcolo la probabilità di accettare la mossa $\gamma(\direct)=min\left\lbrace 1,\frac{\ph(x^t)q(X_t | x^t)}{\ph(X_t)q(x^t|X_t  )}\right\rbrace$
\State Estraggo una variabile da una distribuzione uniforme $u\sim U(0,1)$
\If {$u \leq \gamma(x^t | X_t)$}
 \State La catena transita nel nuovo stato proposto $X_{t+1}=x^t$
\Else
\State La catena rimane nel vecchio stato $X_{t+1}=X_t$
\EndIf
\EndFor
\end{algorithmic}

